\documentclass{bioinfo}
\copyrightyear{2015} \pubyear{2015}

\access{Advance Access Publication Date: Day Month Year}
\appnotes{Manuscript Category}

\begin{document}
\firstpage{1}

\subtitle{Subject Section}

\title[mulTree]{mulTree: using phylogenetic uncertainty in comparative methods  }
\author[Guillerme \textit{et~al}.]{Thomas Guillerme\,$^{\text{\sfb 1,}*}$, Natalie Cooper\,$^{\text{\sfb 2}}$ and Kevin Healy\,$^{\text{\sfb 3,}}$}
\address{$^{\text{\sf 1}}$Silwood Park Campus, Department of Life Sciences, Imperial College London, Buckhurst Road, Ascot SL5 7PY, United Kingdom and \\
$^{\text{\sf 2}}$ Department of Life Sciences, Natural History Museum, Cromwell Road, London, SW7 5BD, United Kingdom and \\
$^{\text{\sf 3}}$ Department of Zoology, School of Natural Sciences, University of Dublin, Trinity College, Dublin, Ireland,
Country.}

\corresp{$^\ast$To whom correspondence should be addressed.}

\history{Received on XXXXX; revised on XXXXX; accepted on XXXXX}

\editor{Associate Editor: XXXXXXX}

\abstract{
\textbf{Motivation:}\\
\textbf{Results:}\\

\textbf{Contact:} \href{t.guillerme@imperial.ac.uk}{t.guillerme@imperial.ac.uk}\\
\textbf{Supplementary information:} Supplementary data are available at \textit{Bioinformatics}
online.}

\maketitle

% Authors instructions: http://www.oxfordjournals.org/our_journals/bioinformatics/for_authors/general.html
% Application Notes (up to 2 pages; this is approx. 1300 words or 1000 words plus one figure) Applications Notes are short descriptions of novel software or new algorithm implementations, databases and network services (web servers, and interfaces). Software or data must be freely available to non-commercial users. Availability and Implementation must be clearly stated in the article. Authors must also ensure that the software is available for a full TWO YEARS following publication. Web services must not require mandatory registration by the user. Additional Supplementary data can be published online-only by the journal. This supplementary material should be referred to in the abstract of the Application Note. If describing software, the software should run under nearly all conditions on a wide range of machines. Web servers should not be browser specific. Application Notes must not describe trivial utilities, nor involve significant investment of time for the user to install. The name of the application should be included in the title. 

\section{Introduction}

%\enlargethispage{12pt}

\section{Approach}



\begin{methods}
\section{Methods}

\end{methods}


\section{Discussion}







%%%%%%%%%%%%%%%%%%%%%%%%%%%%%%%%%%%%%%%%%%%%%%%%%%%%%%%%%%%%%%%%%%%%%%%%%%%%%%%%%%%%%
%
%     please remove the " % " symbol from \centerline{\includegraphics{fig01.eps}}
%     as it may ignore the figures.
%
%%%%%%%%%%%%%%%%%%%%%%%%%%%%%%%%%%%%%%%%%%%%%%%%%%%%%%%%%%%%%%%%%%%%%%%%%%%%%%%%%%%%%%






\section{Conclusion}


\section*{Acknowledgements}

%Thanks early users for testing: Elspeth Kenny, Danny Rojas Martín, Jorge Sanchez Gutierrez, Laura Garrison, Oscar Inostroza

\section*{Funding}

This work has been supported by the... Text Text  Text Text.\vspace*{-12pt}

%\bibliographystyle{natbib}
%\bibliographystyle{achemnat}
%\bibliographystyle{plainnat}
%\bibliographystyle{abbrv}
%\bibliographystyle{bioinformatics}
%
%\bibliographystyle{plain}
%
\bibliography{Document}


% TG: Use the following bib format for submission:
% \begin{thebibliography}{}

% \end{thebibliography}
\end{document}
